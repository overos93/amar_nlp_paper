% Options for packages loaded elsewhere
\PassOptionsToPackage{unicode}{hyperref}
\PassOptionsToPackage{hyphens}{url}
%
\documentclass[
]{article}
\usepackage{amsmath,amssymb}
\usepackage{lmodern}
\usepackage{iftex}
\ifPDFTeX
  \usepackage[T1]{fontenc}
  \usepackage[utf8]{inputenc}
  \usepackage{textcomp} % provide euro and other symbols
\else % if luatex or xetex
  \usepackage{unicode-math}
  \defaultfontfeatures{Scale=MatchLowercase}
  \defaultfontfeatures[\rmfamily]{Ligatures=TeX,Scale=1}
\fi
% Use upquote if available, for straight quotes in verbatim environments
\IfFileExists{upquote.sty}{\usepackage{upquote}}{}
\IfFileExists{microtype.sty}{% use microtype if available
  \usepackage[]{microtype}
  \UseMicrotypeSet[protrusion]{basicmath} % disable protrusion for tt fonts
}{}
\makeatletter
\@ifundefined{KOMAClassName}{% if non-KOMA class
  \IfFileExists{parskip.sty}{%
    \usepackage{parskip}
  }{% else
    \setlength{\parindent}{0pt}
    \setlength{\parskip}{6pt plus 2pt minus 1pt}}
}{% if KOMA class
  \KOMAoptions{parskip=half}}
\makeatother
\usepackage{xcolor}
\IfFileExists{xurl.sty}{\usepackage{xurl}}{} % add URL line breaks if available
\IfFileExists{bookmark.sty}{\usepackage{bookmark}}{\usepackage{hyperref}}
\hypersetup{
  pdftitle={Analysis},
  pdfauthor={Henry Overos},
  hidelinks,
  pdfcreator={LaTeX via pandoc}}
\urlstyle{same} % disable monospaced font for URLs
\usepackage[margin=1in]{geometry}
\usepackage{graphicx}
\makeatletter
\def\maxwidth{\ifdim\Gin@nat@width>\linewidth\linewidth\else\Gin@nat@width\fi}
\def\maxheight{\ifdim\Gin@nat@height>\textheight\textheight\else\Gin@nat@height\fi}
\makeatother
% Scale images if necessary, so that they will not overflow the page
% margins by default, and it is still possible to overwrite the defaults
% using explicit options in \includegraphics[width, height, ...]{}
\setkeys{Gin}{width=\maxwidth,height=\maxheight,keepaspectratio}
% Set default figure placement to htbp
\makeatletter
\def\fps@figure{htbp}
\makeatother
\setlength{\emergencystretch}{3em} % prevent overfull lines
\providecommand{\tightlist}{%
  \setlength{\itemsep}{0pt}\setlength{\parskip}{0pt}}
\setcounter{secnumdepth}{-\maxdimen} % remove section numbering
\newlength{\cslhangindent}
\setlength{\cslhangindent}{1.5em}
\newlength{\csllabelwidth}
\setlength{\csllabelwidth}{3em}
\newlength{\cslentryspacingunit} % times entry-spacing
\setlength{\cslentryspacingunit}{\parskip}
\newenvironment{CSLReferences}[2] % #1 hanging-ident, #2 entry spacing
 {% don't indent paragraphs
  \setlength{\parindent}{0pt}
  % turn on hanging indent if param 1 is 1
  \ifodd #1
  \let\oldpar\par
  \def\par{\hangindent=\cslhangindent\oldpar}
  \fi
  % set entry spacing
  \setlength{\parskip}{#2\cslentryspacingunit}
 }%
 {}
\usepackage{calc}
\newcommand{\CSLBlock}[1]{#1\hfill\break}
\newcommand{\CSLLeftMargin}[1]{\parbox[t]{\csllabelwidth}{#1}}
\newcommand{\CSLRightInline}[1]{\parbox[t]{\linewidth - \csllabelwidth}{#1}\break}
\newcommand{\CSLIndent}[1]{\hspace{\cslhangindent}#1}
\ifLuaTeX
  \usepackage{selnolig}  % disable illegal ligatures
\fi

\title{Analysis}
\author{Henry Overos}
\date{3/16/2022}

\begin{document}
\maketitle

\hypertarget{testing-modeling-of-protest}{%
\section{Testing Modeling of
Protest}\label{testing-modeling-of-protest}}

To demonstrate the validity of the workflow presented in our paper, we
conducted a test of model accuracy on one of the AMAR data's most
valuable variables, protest. The AMAR protest variable identifies the
presence and scale of minority protest for a specific
minority-country-year. Our team used the gold standard of human coded
news articles collected to estimate the overall accuracy of
semi-supervised learning on identifying protest in the news.

\hypertarget{newsmap}{%
\subsection{Newsmap}\label{newsmap}}

The workhorse model for our method is \emph{Newsmap}, a semi-supervised
variation of the \emph{naive Bayes} classifier Watanabe (2018).
\emph{Newsmap} was originally created to identify the geo-location of
news stories but has been shown to be useful in a variety of other
contexts, including the classification of UN speeches Watanabe and Zhou
(2020) and news topics Watanabe (n.d.).

The model takes a dictionary of ``seed words'' based on prior knowledge
of the subject to inform the classification. In the case of AMAR's
protest variable, the model should identify if a text is about protest.

We also separated protest into two categories based on the severity of
the protest event. This was done to mirror the original coding schema of
AMAR, which identified three types of Protests (for a specific
explanation about these types, refer to the Appendix). We separated the
types into two, verbal opposition, or mass resistance. Verbal opposition
refers to a group or representative of a group making a statement of
protest, creating a petition, giving a speech in opposition, or some
other sort of action that signifies disagreement and is meant to draw
attention to an issue. The second category, mass resistance, refers to
actions taken by groups or large numbers of people to visibly or
symbolically highlight an issue. Mass resistance can take the form of
marches, sit-ins, rallies, strikes, blocking streets, and other, similar
forms of action.

To test the effectiveness of the model across specifications, we test
both a binary classifier, which identifies protest/not protest, and a
tertiary classifier which categorizes articles as verbal/mass
protest/not protest.

\hypertarget{seed-word-selection}{%
\subsection{Seed Word Selection}\label{seed-word-selection}}

Researchers selected initial seed-words related to the topic of protest
by examining the AMAR codebook's wording, as well as words found in
books, articles and indices on the topic. Further words were added based
on frequency in the text. We collected the 300 most frequent terms in
the data and manually identified if any could also be included as seed
words in the dictionary. We then used a process of assessing the utility
of frequency-selected words based on the suggestions on the topic found
in Watanabe and Zhou (2020). The resulting dictionary appears in table
1.

\begin{table}[!htb]
\centering
\caption{Protest Dictionary for Multi-Class Protest Classification}
\begin{tabular}{l|l}
\hline
\textbf{Variable} & \textbf{Words} \\ \hline
Verbal Opposition & aware*, petition*, outcr*, support*, urge*, lobby*, demand*, \\
 & calling, comment*, voice* \\
Mass Demonstration & disobedience, property, demonstrat*, unrest, violence
\end{tabular}
\end{table}

\begin{table}[!htb]
\centering
\caption{Protest Dictionary for Binary Protest Classification}
\begin{tabular}{l|l}
\hline
\textbf{Variable} & \textbf{Words} \\ \hline
Protest: & aware*, petition*, outcr*, support*, urge*, lobby*, demand*, \\
 & comment*, voice*, block*, property, small, damage*, cancel*, \\
 & resign*, violence, death, unrest, street*, organize*
\end{tabular}
\end{table}

\hypertarget{human-coded-test-set}{%
\subsection{Human-Coded Test Set}\label{human-coded-test-set}}

Creating the test set was a two step process so that we could compare
the accuracy of modeling across whole articles as well as at the
sentence level. An entire article may cover multiple topics, protest
being one of them. Sentences, however, are much more likely to be
related to a single topic and the topics will more likely continue from
one sentence to the next. We hypothesized that a change in the unit of
analysis from articles to sentences would increase the model's overall
classification accuracy. As such, a team of annotators read 200 U.S.
news articles from the year 2020 and identified whether each article was
related to protest. These researchers also rated the protest by `type',
corresponding to the AMAR codebook's categories (see Appendix).

We then took a sample of 20 articles identified as being about protest
by annotators and broke them into sentences. A separate team of
annotators then labelled the sentences in each article, identifying the
presence of and level of protest. We added extra instructions for the
sentence-level annotation, allowing the annotators to categorize
hard-to-annotate sentences \emph{in the context of the preceding and
following} sentences. This is possible because \emph{Newsmap} allows for
smoothing, which, in practice, means that the category of the
surrounding texts affects the likelihood that a specific text is also
that category. The 20 articles converted into 621 sentences used in
annotation, greatly increasing the \emph{N} for analysis in the sample.

\hypertarget{sentence-vs.-article-results-comparison}{%
\subsection{Sentence vs.~Article Results
Comparison}\label{sentence-vs.-article-results-comparison}}

We present the results of the \emph{Newsmap} models, comparing the
scores of both the binary and multi-class models. We also show the
difference in model results based on whether we conducted the analysis
at the sentence or article level.

Table 3 shows the precision, recall, and F1 values for the binary
classification of protest with Newsmap. The results suggest that the
model trained on the same articles, but broken up into sentences, is
more accurate in predicting the correct outcomes in the human-coded test
data.

Table 4 also shows the same metrics but applied to the multi-class
model. These results present different outcomes both for sentences and
articles but between classes. The sentence-trained model's highest
scores go to the \emph{not protest} classification. This is likely
because, in the data, there are more sentences \emph{not} about protest
than there are about it. In the article-trained model, \emph{not
protest} drops to an F1 value of 0.41, although the precision score is
higher.

\begin{table}[!htb]
\centering
\caption{Binary Classification Model Results}
\begin{tabular}{l|lll|lll}
\hline
 & \multicolumn{3}{c|}{Sentences} & \multicolumn{3}{c}{Articles} \\ \cline{2-7}
 & \multicolumn{1}{l|}{P} & \multicolumn{1}{l|}{R} & F1 & \multicolumn{1}{l|}{P} & \multicolumn{1}{l|}{R} & F1 \\ \cline{2-7} 
Not Protest/Protest & \multicolumn{1}{l|}{0.76} & \multicolumn{1}{l|}{0.81} & 0.79 & \multicolumn{1}{l|}{.83} & \multicolumn{1}{l|}{0.24} & 0.37 \\ \hline
\end{tabular}
\end{table}

\begin{table}[!htb]
\centering
\caption{Multi-Class Classification Model Results}
\begin{tabular}{l|lll|lll}
\hline
 & \multicolumn{3}{c|}{Sentences} & \multicolumn{3}{c}{Articles} \\ \cline{2-7} 
 & \multicolumn{1}{c|}{P} & \multicolumn{1}{c|}{R} & \multicolumn{1}{c|}{F1} & \multicolumn{1}{c|}{P} & \multicolumn{1}{c|}{R} & \multicolumn{1}{c}{F1} \\ \cline{2-7} 
Not Protest & \multicolumn{1}{l|}{0.76} & \multicolumn{1}{l|}{0.74} & 0.75 & \multicolumn{1}{l|}{0.83} & \multicolumn{1}{l|}{0.33} & 0.41 \\
Verbal Opp. & \multicolumn{1}{l|}{0.11} & \multicolumn{1}{l|}{0.076} & 0.09 & \multicolumn{1}{l|}{0.065} & \multicolumn{1}{l|}{0.61} & 0.12 \\
Mass Protest & \multicolumn{1}{l|}{0.18} & \multicolumn{1}{l|}{0.34} & 0.24 & \multicolumn{1}{l|}{0.84} & \multicolumn{1}{l|}{0.55} & 0.66 \\ \hline
\end{tabular}
\end{table}

\hypertarget{references}{%
\subsection*{References}\label{references}}
\addcontentsline{toc}{subsection}{References}

\hypertarget{refs}{}
\begin{CSLReferences}{1}{0}
\leavevmode\vadjust pre{\hypertarget{ref-watanabe_newsmap_2018}{}}%
Watanabe, Kohei. 2018. {``Newsmap: {A} Semi-Supervised Approach to
Geographical News Classification.''} \emph{Digital Journalism} 6 (3):
294--309. \url{https://doi.org/gg6v46}.

\leavevmode\vadjust pre{\hypertarget{ref-watanabe_conspiracist_}{}}%
---------. n.d. {``Conspiracist Propaganda: {How Russia} Promotes
Anti-Establishment Sentiment Online?''} 32.

\leavevmode\vadjust pre{\hypertarget{ref-watanabe_theory-driven_2020}{}}%
Watanabe, Kohei, and Yuan Zhou. 2020. {``Theory-{Driven Analysis} of
{Large Corpora}: {Semisupervised Topic Classification} of the {UN
Speeches}:''} \emph{Social Science Computer Review}, February.
\url{https://doi.org/ggr99p}.

\end{CSLReferences}

\end{document}
